
%%%%% Définition paramètres documents

\documentclass[a4paper, 12pt]{report} % Initilisation document

% Paramètres texte

\usepackage[utf8]{inputenc} % Français avec caractères spéciaux
\usepackage[french]{babel} % Français pour la biographie
\usepackage[T1]{fontenc} % Polices d'écriture

% Mise en page

\usepackage{bookmark}
\usepackage[left=2cm, right=2cm, top=2cm, bottom=2cm]{geometry} % Dimensions marges
\usepackage{indentfirst} 
\usepackage{underscore}
\usepackage{nopageno} % Supprimer la pagination d'une page
\usepackage{float, caption}	% Positionnement et légende des images
\usepackage{fancyhdr} % Marges
\usepackage{calc, ifthen, xspace} % Redéfinition des espaces et distances
\usepackage{titlesec}

% Paramètres mathématiques

\usepackage{amsmath, amsthm, amssymb, amsfonts} % Polices et symboles mathématiques sophistiqués

% Paramètres graphiques

\usepackage{graphicx} % Inclusion images
\usepackage{epic,eepic}	% Positionnement image de garde
\usepackage{wrapfig} % Images sur le coté dans le texte

% Autres

\usepackage{hyperref} % Liens croisés fichier PDF
\usepackage{color} % Couleur codes informatiques
\usepackage{lastpage} % Références aux pages
\usepackage{endnotes} % Notes de fin

% Identation automatique

\setlength{\parindent}{0pt} % Supression des alinéas automatiques par défaut, plus manipulable

% Réduire les espacements des titres de chapitre, trop importants par défaut

\titleformat{\chapter}[display]{\normalfont\huge\bfseries}{\chaptertitlename \ \thechapter}{20pt}{\Huge}   
\titlespacing*{\chapter}{0pt}{-25pt}{15pt} % {alinéa}{au dessus}{en dessous}

% Paramètres hyperliens

\hypersetup{
% dvips,
% backref=true,
% pagebackref=true, % Bibliographies
% hyperindex=true, % Index.
colorlinks = true, % Couleur hyperliens
breaklinks = true, % Retour à la ligne dans les hyperliens trop longs
urlcolor = blue, % Couleur des hyperliens
linkcolor = black, % Couleur des liens internes
% bookmarks=blue, % Signets pour Acrobat
bookmarksopen = true}

\makeatletter
\g@addto@macro{\UrlBreaks}{\do\-\do\.}
\makeatother

% Paramètres de différentes couleurs avec {nom_couleur}{type_encodage}{proportions}

\definecolor{hellgelb}{rgb}{1,1,0.8}
\definecolor{colKeys}{rgb}{0,0,1}
\definecolor{colIdentifier}{rgb}{0,0,0}
\definecolor{colComments}{rgb}{0,0.5,0}
\definecolor{colString}{rgb}{0.62,0.12,0.94}
\definecolor{INSA_GM}{cmyk}{0.6,0,0,0}
\definecolor{INSA_GRIS}{cmyk}{0.7,0.6,0.5,0.3}
\definecolor{INSA_BLEU}{cmyk}{1,0.9,0.1,0}

% Définition de commandes compliquées pour déclarer un nom de projet

\newcommand{\insertrefproj}[1]{}
\newcommand{\refproj}[1]{\renewcommand{\insertrefproj}{\textbf{\color{INSA_GRIS}#1}}}

% Style de page

\pagestyle{fancy}

\fancypagestyle{courant}{
	\fancyhf{}
    \setlength{\headheight}{27pt}
    \fancyhead[L]{\raisebox{-2mm}{\includegraphics[width=30mm]{Images/Logo INSA.png}}}
\fancyhead[C]{}
\fancyhead[R]{\color{INSA_GRIS}\thepage}%\scshape\leftmark}
\fancyfoot[L]{\insertrefproj}
\fancyfoot[R]{}
%\fancyfoot[CE,CO]{\color{INSA_GRIS}\thepage}
\renewcommand{\headrulewidth}{0pt}
\renewcommand{\footrulewidth}{0.2pt}
}

\fancypagestyle{special}{%
  \pagestyle{courant}
  \fancyfoot{}
  %\addtolength{\footskip}{-20pt}
%  \setlength{\footskip}{0pt}
%  \fancyfoot[C]{}
\renewcommand{\footrulewidth}{0pt}
}

\fancypagestyle{plain}{%
  \fancyhf{}%
  \pagestyle{courant}
}

% 

\newcommand{\Scilab}{
	\lstset{
	language=Scilab,
	float=hbp,
	basicstyle=\ttfamily\small,
	identifierstyle=\color{colIdentifier},
	keywordstyle=\bf \color{colKeys},
	stringstyle=\color{colString},
	commentstyle=\color{colComments},
	columns=flexible,
	tabsize=5,
	frame=single,
	%frame=shadowbox,
	rulesepcolor=\color[gray]{0.5},
	extendedchars=true,
	showspaces=false,
	showstringspaces=false,
	numbers=left,
	stepnumber=5,
	firstnumber=1,
	numberstyle=\tiny,
	breaklines=true,
	%backgroundcolor=\color{hellgelb},
	captionpos=b,
	}
}

% Paramètres projet

\title{Projet de P6}
\refproj{STPI/P6/2024 - 39}
\author{}
\date{}

%%%%% Début d'édition du document

\begin{document}

%%% Page de présentation

\pagenumbering{Alph} % Numérotation avec lettre majuscule
\thispagestyle{empty} % Type page vide intiallement

\vspace{4cm}

\begin{picture}(0,0) % Hauteur, Largeur
	
\put(80,-465){\includegraphics[scale=1]{Images/Image Garde.png}} % Image de garde centrée horizontalement
\put(0,-20){\includegraphics[width=0.4\textwidth]{Images/Logo INSA.png}} % Logo INSA haut-gauche
\put(240,0){{\begin{minipage}{12cm}\centering \Large % Caractéristiques projet haut droit
	\textbf{Projet de Physique P6} \\ 
	\textbf{STPI/P6/2024 - 39}\end{minipage}}}
\put(-20,-150){{\begin{minipage}{\textwidth}\centering \Huge % Titre princpal rapport
	\textbf{Modélisation de l'impact des gaz à effet de serre}\end{minipage}}}
% Quelques changements pour illustrer
\newsavebox{\noms}
\savebox{\noms}(300, 300)[tl]{
\put(0,98){\color{INSA_GRIS}{\begin{minipage}{6cm} 
	\textbf{Étudiants :} \end{minipage}}} % Étudiants par ordre alphabétique de nom
\put(0,60){\color{INSA_GRIS}{\begin{minipage}{6cm}
	Romane LANERES \\ Anouk PETITGAS \\ Arthur SARRAU \end{minipage}}}
\put(185,60){\color{INSA_GRIS}{\begin{minipage}{6cm}
	Clara MÉLINE \\ Tom PHILIPPE \\ Nina ZEDDOUN \end{minipage}}}

\put(0,){\color{INSA_GRIS}\begin{minipage}{9cm}   
	\textbf{Enseignant-responsable du projet :} \\ Samuel PAILLAT \end{minipage}}}

\put(100,-850){\usebox{\noms}}

\end{picture}

%%% Page complètement vierge

\newpage
\pagenumbering{arabic}
\setcounter{page}{1}
\thispagestyle{empty}
\null

%%% Page descriptions cractéristiques principales projet

\newpage
\pagestyle{special}

\chapter*{Fiche projet} % * Garde la même mise en page qu'un chapitre mais ne le numérote pas
\addcontentsline{toc}{chapter}{Fiche projet} % Ajout dans le sommaire, possiblement renommé comme souhaité

\textbf{Date de remise du rapport :} 13/06/2024 \vspace{\baselineskip}
% \vspace{\baselineskip} permet de sauter des lignes dans le code (pour la lisibilité)
% en évitant tout message d'erreur de la forme "Underfull \hbox (badness 10000)"

\textbf{Référence du projet:} STPI/P6/2024 - 39 \vspace{\baselineskip}

\textbf{Intitulé du projet :} Modélisation de l'impact des gaz à effet de serre \vspace{\baselineskip}

\textbf{Type de projet :} Modélisation numérique \vspace{\baselineskip}

\textbf{Objectifs du projet :} \vspace{\baselineskip} 

Ce projet a pour vocation de développer un modèle numérique
décrivant les mécanismes de régulation thermique de 
l'atmosphère terrestre. Premièrement, nous allons réaliser
une étude à travers quelques calculs en ordre de grandeurs afin
d'identifier les différents paramètres d'influence de notre modèle.
L'objectif est de mettre au point un modèle qui soit modulable
en termes de paramètres et de complexité. Cette dernière est 
ajustée par le traitement de différents gaz à effet de serre 
et de divers modèles de température et de pression. 
L'ensemble des modélisations numériques aspire à proposer 
une version améliorée du travail réalisé par le 
vulgarisateur scientifique David Louapre. À travers ce projet, 
nous cherchons à reproduire au mieux le 
phénomène d'équilibre thermique de la Terre en considérant 
les flux principaux et en quantifiant précisément l'impact
des gaz à effet de serre dans le système atmosphérique. \vspace{\baselineskip}

\textbf{Mots-clefs du projet :} Transfert thermique, Corps noir, Effet de serre, Bilan radiatif 
\vspace{\baselineskip} 

\vfill

% Coordonnées INSA Rouen par défaut

\begin{center}
	\scshape Institut National des Sciences Appliquées de Rouen \\
	Département Sciences et Techniques Pour l'Ingénieur \\
	685 Avenue de l'Université BP 08- 76801 Saint-Etienne-du-Rouvray \\ Tél : 33 2 32 95 66 21 - Fax : 33 2 32 95 66 31
\end{center}

%%% Remerciements

\newpage
\chapter*{Remerciements} 
\addcontentsline{toc}{chapter}{Remerciements}

\setlength{\parindent}{30pt}

\indent Nous souhaiterions avant tout remercier Samuel PAILLAT, notre
coordinateur et enseignant encadrant, qui a su nous aiguiller
et nous apporter son aide et ses connaissances tout au longs
de ce projet. \vspace{\baselineskip}

	Aussi nous aimerions nous porter reconnaissants 
envers le physicien médaillé de la médiation scientifique
du CNRS: David Louapre. Ces travaux sur l'effet de serre 
ont été une grande source d'inspiration en tant que point
de départ pour notre recherche \endnote{DAVID LOUAPRE, Article et vidéo,
\textit{"La saturation de l’effet de serre (ou pas)"}, 06/10/2023:
\\ \url{https://scienceetonnante.com/2023/10/06/la-saturation-de-leffet-de-serre/}}. 
\vspace{\baselineskip}

\indent Nous saluons également les divers organismes ayant 
mis à disposition des bases de données en open source. Nous
en avons extrait des paramètres physiques sans lesqueles 
nos modèles auraient manqué de fiabilité. \vspace{\baselineskip}

\indent Enfin, nous aimerions remercier l'organisation de 
l'INSA plus généralement pour nous avoir implémenté l'EC de P6
qui nous a permis de réaliser notre premier projet
scientifique appliqué dans des conditions de collaboration
collective proches de celle milieu de l'ingénierie. 
En effet depuis le début de ce semestre nous avons tous commencé à 
nous pré-spécialiser dans un domaine de l'ingénierie. À travers 
ce projet chacun est amené à mettre en commun ses savoirs et 
compétences acquis pendant ces derniers mois, ce qui s'avère est 
enrichissant pour tout le groupe.

\vfill

%%% Table des matières 

\newpage
\pagestyle{courant} 
	\setcounter{tocdepth}{2} % Règle la profondeur maximale de la table des matières
	\tableofcontents % Affichage table des matières

%%% Notations et Acronymes

\newpage
\chapter*{Notations} 
\addcontentsline{toc}{chapter}{Notations}

\begin{description}
	\subsubsection*{Acronymes}
	\item[GES:] Gaz à Effet de Serre
    \item[CN:] Corps Noir
    
	\subsubsection*{Grandeurs \footnote{En unités usuelles}}
    \item[\boldmath{$\Phi$}:] Flux  ($W$)
    \item[\boldmath{$\phi$}:] Densité de flux \footnote{Dans ce rapport il décrira exclusivement la puissance rayonnée par unité de surface} ($W.m^{-2}$)
    \item[\boldmath{$M^{0}$}:] Émittance totale d'un corps noir ($W.m^{-2}$)
    \item[\boldmath{$M^{0}_{\lambda,T}$}:] Émittance d'un corps noir ($W.m^{-3}$)
    \item[\boldmath{$\epsilon$}:] Émissivité ($sans$ $unit\acute e$)
    \item[\boldmath{$\tau_{CO_2}$}:] Coefficient de transmission du $CO_2$ ($sans$ $unit\acute e$)
	\item[\boldmath{$k_{abs}$}:]: coefficient d'absorption obtenue avec la base de donnée HITRAN ($m^2.molec^{-1}$)
    \item[\boldmath{$n_{CO2}(z)$}:] Quantité de matière de CO2 en fonction de l'altitude ($molec.m^{-3}$)
    \item[\boldmath{$\lambda$}:] longueur d'onde ($m$)
    \item[\boldmath{$P$}:] Pression ($Pa$)
    \item[\boldmath{$V$}:] Volume ($m^3$)
	\item[\boldmath{$n$}:] Quantité de matière ($mol$)
	\item[\boldmath{$T$}:] Température ($K$)
 	\item[\boldmath{$M$}:] Masse molaire ($kg.mol^{-1}$)
	
	\subsubsection*{Constantes \footnote{En Unités du Système International (USI) à 3 CS près}}
	\item[\boldmath{$R$}:] Constante universelle des gaz parfaits = $8,314$ $kg.m^2.mol^{-1}.K^{-1}.s^{-2}$
	\item[\boldmath{$\sigma$}:] Constante de Stefan-Boltzmann = $5,67.10^{-8}$ $W.m^{-2}.K^{4}$
	\item[\boldmath{$h$}:] Constante de Plank = $6,63.10^{-34}$ $kg.m^2.s^{-1}$
 	\item[\boldmath{$k_B$}:] Constante de Boltzmann = $1,38.10^{-23}$ $kg.m^2.s^{-2}.K^{-1}$
  	\item[\boldmath{$g$}:] Constante gravitationelle terrestre = $9,81$ $m.s^{-2}$
   	\item[\boldmath{$c_0$}:] Célérité de la lumière = $2,99^8$ $m.s^{-1}$
    \item[\boldmath{$C_1$}:] Constante de Planck 1 = $1,19.10^{-16}$ $kg.s^{-3}.m^{-4}$
    \item[\boldmath{$C_2$}:] Constante de Planck 2 = $1,44.10^{-2}$ $m.K$
    \item[\boldmath{$T_s$}:] Température moyenne du Soleil = $5 772$ $K$
    \item[\boldmath{$T_T$}:] Température moyenne de la Terre = $288,15$ $K$
	\item[\boldmath{$\bar{a_T}$}:] Albédo terrestre moyen = $0,31$
    \item[\boldmath{$R_T$}:] Rayon de la Terre = $6 371$ $km$

\end{description}

%%% Introduction

\newpage
\chapter*{Introduction}				
\addcontentsline{toc}{chapter}{Introduction}

\indent Le réchauffement climatique, dû à l'effet de serre, est sans aucun doute la plus grande problématique 
scientifique du siècle actuel. Elle regroupe une multitude d'enjeux mêlant environnement,
climat, industries et sociétés humaines. \vspace{\baselineskip}

\indent Cependant, il semblerait que ce phénomène physique, vraisemblablement assez simple 
en apparence, s'avère bien plus complexe en réalité. Si l'on se penche dans un premier temps sur 
l'étymologie de ce terme, inventé par le scientifique français Joseph Fourier
\endnote{JOSEPH FOURIER, Article, 
\textit{"Mémoire sur les températures du globe terrestre et des espaces planétaires}, 
Mémoires de l'Académie royale des sciences de l'Institut de France, vol.7,  p.569-604, 1827:
\\ \url{https://gallica.bnf.fr/ark:/12148/bpt6k32227/f814.item}},
on recontre déjà une première confusion notable. En effet l'effet de serre 
tel qu'il opère dans les couches atmosphériques est majoritairement lié au jeu d'absorption, d'émission 
et de transmission de rayonnements. Tandis que les serres misent plutôt sur l'utilisation d'une enceinte 
fermée hermétique qui supprime l'effet de convection de l'air, ce qui a pour conséquence une élévation 
de température. En bref, il faut garder en tête que cette analogie comporte des limites. \vspace{\baselineskip}

\indent De ce fait, c'est dans ce cadre que s'inscrit notre projet, qui a pour but de quantifier et d'étudier
l'impact des gaz à effet de serre avec des outils de deuxième année de cursus ingénieur. L'objectif est 
de construire petit à petit un modèle, avec de plus en plus de paramètres, et qui tende à se rapprocher
d'une description fidèle de l'influence des gaz à effet de serre. Au passage il est intéressant de souligner
que nous traiterons ce phénomène indépendamment de son origine, qu'elle soit naturelle ou anthropique. \vspace{\baselineskip}

\indent Tout d'abord nous réaliserons quelques calculs initiaux de flux sans aucun GES.
Ensuite nous introduirons le dioxyde de carbone $CO_2$ et étudierons son impact avec une température de
système uniforme. Puis, si le temps nous le permet, on considérera d'autres GES et on ferra évoluer leur caractéristiques
en fonction de modèles atmosphériques de température, altitude et pression. \vspace{\baselineskip}

\indent Dans l'ensemble, ce projet se base sur la physique des transferts thermiques, 
mais fait également appel à des compétences en informatique et mathématiques.
En effet notre modèle se fonde sur plusieurs paramètres qu'il sera nécessaire
d'ajuster selon nos besoins, ce qui requiert d'implémenter tous ces calculs
numériquement. \vspace{\baselineskip}

\indent Pour ce qui est des outils, nous avons codé l'ensemble de nos algorithmes en Python 3.12 et avons misé
la popularité de ce langage pour utiliser des bibliothèques de physique et de mathématiques libres d'accès
permettant d'avoir accès à des outils absolument fondamentaux: calcul intégral numérique, 
base de donnée de taux d'absorptions de gaz à effet de serre, graphiques... Sinon, nous avons utilisé du \LaTeX
pour la rédaction de ce rapport, qui est un logiciel parfaitement adapté aux études dans le milieu scientifique.
La prise en main s'est faite progressivement pour chaque membre du groupe, mais au final, nous sommes tous parvenus
à maîtriser les bases fondamentales et rudiments de cet outil de traitement de texte à première vue bien complexe.

%%% Méthodologie, organisation du travail

\chapter{Méthodologie, organisation du travail}

\begin{itemize}
\item Description de l'organisation adoptée pour le déroulement du travail
\item Organigramme des tâches réalisées et des étudiants concernés
\end{itemize}

%%% Travail réalisé et résultats

\chapter{Travail réalisé et résultats}

\section{L'effet de serre}

\subsection{Explication du phénomène de l'effet de serre}

	L'effet de serre est un processus naturel qui se produit 
avec le rayonnement entre la Terre et le Soleil. Tout d'abord, 
le Soleil émet un rayonnement vers la Terre, l'atmosphère 
laisse passer une partie de ce rayonnement solaire. 
Réchauffé, le corps terrestre, étant un corps noir, émet un rayonnement infrarouge. 
Ce flux est alors renvoyé vers l'atmosphère et une partie est absorbée par les gaz présents, appelés gaz à effet de serre. 
Le reste du flux est envoyé dans l'espace.

\begin{center}
    \includegraphics[scale=0.35]{Images/schemaflux.png} 
    
    Figure 1 : Bilan radiatif terrestre
\end{center} 

Dans le but de comprendre ce phénomène plus en détails, nous allons maintenant expliciter 
différentes hypothèses. Premièrement nous allons considérer que les différents systèmes impliqués 
dans le phénomène d’effet de serre sont des corps noirs. Par la suite, nous allons confronter 
deux modèles : un premier sans l'effet de serre sur la Terre et un second avec.
\vspace{\baselineskip}

Afin de modéliser ces deux modèles nous allons préciser les différents transferts thermiques 
existant entre le soleil, la Terre et l’espace. Le seul échange thermique possible entre le Soleil
 et la Terre est le rayonnement.
\vspace{\baselineskip}

Tout d’abord, nous allons commencer par créer un modèle simple où la température de la surface de 
la Terre est égale à la température de l’atmosphère, et donc de ses couches supérieures. Nous savons 
que l’atmosphère émet un rayonnement selon la loi de Planck qui dépend de la température. Par 
conséquent, les températures de la surface de la terre et de l’atmosphère étant égales dans ce 
premier modèle, ce qui est absorbé par l’atmosphère est ré-émis exactement de la même manière.
Ainsi, si l'on modélise ce phénomène afin d’obtenir le spectre au sommet de l’atmosphère en mettant 
en entrée ……. on obtiendrait les courbes suivantes :
\vspace{\baselineskip}

(2 modèles)
\vspace{\baselineskip}


En comparant ces deux courbes, nous pouvons voir que le spectre au sommet de l’atmosphère est le 
même avec ou sans la présence de gaz à effet de serre. La seule différence étant l’origine du 
rayonnement qui ne provient que de la surface de la terre pour celui sans effet de serre et de la 
surface de la terre et des couches de l'atmosphère (GES) pour celui avec l’effet de serre. 
Ainsi, si la température était uniforme on n’aurait pas d’impact de gaz à effet de serre.
Or l’atmosphère est en réalité composée de plusieurs couches de température différentes. Chacune 
d’entre elles émettent donc un rayonnement propre selon la loi de Planck. 
\vspace{\baselineskip}

Ainsi, ce qui est absorbé par l’atmosphère n’est pas ré-émis exactement de la même manière.
La différence de température entre la surface de la terre et les couches de l’atmosphère implique 
la présence d’un effet de serre.
\vspace{\baselineskip}

Nous constatons donc que ce phénomène d'effet de serre n'existe 
que lorsque que la température varie dans l'atmosphère en 
fonction de l'altitude. Ainsi si la température était uniforme 
dans l'atmosphère ce phénomène n'existerait pas. \vspace{\baselineskip}

	C'est l'absorption du rayonnement par les GES qui va 
provoqué un réchauffement terrestre. Ainsi, ce phénomène 
créé un réchauffement global de la Terre permettant à la 
température moyenne de s'élever à 15°C au lieu de -18°C 
(sans GES). 

\subsection{Définition des GES}
Un gaz à effet de serre est un gaz présent dans l'atmosphère qui 
absorbe une partie des rayonnements (rayons infrarouges) 
reçue par le soleil. Ces gaz sont d'origine naturelle 
(vapeur d'eau, ozone...) et/ou anthropique 
(issues des activités humaines) comme par exemple le dioxyde 
de carbone ($CO_2$), le méthane ($CH_4$), le protoxyde d'azote
($N_2O$) et les gaz fluorés (HFC) \endnote{JEAN-MARC JANCOVICI, Article, 
\textit{"Quels sont les gaz à effet de serre ?"}, 01/08/2007:
\\ \url{https://jancovici.com/changement-climatique/gaz-a-effet-de-serre-et-cycle-du-carbone/quels
-sont-les-gaz-a-effet-de-serre-quels-sont-leurs-contribution-a-leffet-de-serre/}}. \vspace{\baselineskip}

% mettre schema flux radiatif 

\section{Bilan radiatif terrestre}
\subsection{Définitions du corps noir et de l'émittance}

\subsubsection{Le corps noir}
Un corps noir est un objet idéal qui permet d'évaluer le flux thermique maximum
que peut rayonner un corps, en fonction de sa température. Ainsi un corps noir 
va émettre autant de rayonnement qu'il en absorbe. On peut donc dire que son 
emmitance $ \epsilon$  est égale à 1. 

\subsubsection{L'émittance} 
\noindent L'émittance spéctrique d'un corps noir est évalué grâce à la loi de Planck 
(en $W.m^{-2}$): \vspace{\baselineskip}

\begin{center}
$M^{0}_\lambda,_T= \frac{\pi C_1}{\lambda^5} \biggl(\exp(\frac{C_2}{\lambda T}) - 1\biggr)^{-1}$ 

$C_1$ = 1,19 $\cdot 10^{8}$ $W.m^{-2}.\mu m^{-1}$\
$C_2$ = 14400 $\mu m.K$ \\
\end{center}
Et ainsi l'exitance totale d'un corps noir (en $W.m^{-2}$) est obtenue avec la formule suivante:

\hfil $M^{0}$ = $\int_{0}^{\infty} M^{0}_\lambda d\lambda$ = $\sigma T^{4}$ 

\par

\subsection{Bilan radiatif sans effet de serre} 

Le Soleil émet des rayonnements dans le domaine visible et 
dans le domaine infrarouge. C'est un corps noir, c'est à dire
qu'il absorbe tout le rayonnement qu'il reçoit et le réémet 
parfaitement. De plus , on peut dire que son émissivité est 
égale à 1 d'après les hypothèses du corps noir vues précédemment. Il suit donc la loi de Planck et son spectre d'émission en fonction de 
la longueur d'onde est le suivant : 
\begin{center}
    \includegraphics[scale=0.85]{Images/Spectre Soleil.png}
    
    Figure 2 : Spectre du soleil, intensité du rayonnement solaire en fonction de la longueur d’onde (Brilliant.org)  
\end{center} \vspace{\baselineskip}

\noindent On peut alors calculer ce qu'il émet en rayonnant de 2 manières: \vspace{\baselineskip}

1. Avec l'expression du flux radiatif :
\begin{center}
$\Phi_{ray}$ = $\int_S \epsilon \sigma T^{4} dS$    
\end{center}
\begin{center}
$\Phi_{ray}$ = $\epsilon \sigma T^{4} S$    
\end{center}
avec T=5772 K la température du Soleil, l'émissivité 
$\epsilon = 1$ 
et sa surface $S=4 \pi R^{2}$, sachant que $R= 6,96 \times 10^{8}$ m 
\vspace{\baselineskip}

\noindent On trouve : \vspace{\baselineskip}
$\Phi_{ray}$ = $3,83 \times 10^{26}$ W 

2. Numériquement : \vspace{\baselineskip}

En effet évaluer le flux émit revient à calculer l'aire sous la courbe de son spectre d'émission. (voir Spectre du soleil et insérer 
le code pour faire l'intégrale sous la courbe, on doit retrouver le même phi ray) \vspace{\baselineskip}
\newline On trouve :
$\Phi_{ray}$ = $3,83 \times 10^{26}$ W \vspace{\baselineskip}

Ainsi avec ces deux méthodes, on trouve que le rayonnement émis par le Soleil au niveau de son sol
est de 174 PW  soit $1,74.10^{17}$ W \vspace{\baselineskip}
\vspace{\baselineskip}
\vspace{\baselineskip}


Cependant, comme le Soleil rayonne dans toutes les directions de l’espace, seul une petite partie de ce rayonnement arrive jusqu’à la surface de la Terre. 
Tout d’abord, la distance qui nous sépare du soleil joue un rôle primordial dans cette réduction. 
\newline Elle se calcul comme suit :
\begin{center}
$Distance$ = $\frac{\pi R_T^{2}}{4 \pi D_{(S-T)}^{2}}$    
\end{center}

Le second facteur de diminution s’explique par le phénomène d’albédo : en effet, 30\% du rayonnement qui parvient jusqu’à la Terre est réfléchis par notre planète, donc seulement 70\% du rayonnement est absorbé. 
\newline On peut alors calculer le flux absorbé par la Terre : 
\begin{center}
$\Phi_{abs,Terre}$ = $0,7 \times \Phi_{ray} \times Distance$    
\end{center}
\begin{center}
$\Phi_{abs,Terre}$ = $121$ PW    
\end{center}

En divisant par la surface de la Terre, 
on obtient le flux surfacique absorbé par la Terre pour 1m² :
\begin{center}
$\phi_{abs,Terre}$ = $\frac {121 \times 10^{15}}{4 \pi R_T^{2}}$
= 235 $W.m^{-2}$
\end{center}

\par 
\vspace{\baselineskip}
\vspace{\baselineskip}

Retrouvons maintenant la température à sa surface.
On réalise un bilan thermique sur le système suivant : 
la Terre, sans son atmosphère
\begin{center}
    Production = Échanges + Stockage 
\end{center} \vspace{\baselineskip}

Tout d'abord, la Terre produit de la chaleur grâce au processus de géothermie. En effet, l’activité radioactive au centre de la Terre qui correspond à la désintégration naturelle des atomes présents crée de la chaleur. 
Cependant, cette production thermique est négligeable par rapport à la chaleur apportée par le Soleil. 

\noindent On considère donc que le terme de production thermique dans notre bilan est nul. \vspace{\baselineskip}

De plus, à l’équilibre thermique, l’état est stationnaire, ce qui signifie que la température ne varie plus dans le temps et donc que notre système ne stocke pas d’énergie thermique. Le terme de stockage est nul.

\noindent On obtient alors :
\begin{center}
    Echanges = 0
\end{center}

\begin{center}
    $\Leftrightarrow \Phi_{emis} - \Phi_{absorbe}  = 0$
\end{center}

\begin{center}
    $\Leftrightarrow \Phi_{emis} = \Phi_{absorbe}$
\end{center} \vspace{\baselineskip}

On retrouve que la Terre sans son atmosphère est un corps noir, d’émissivité 1. En effet, tout le rayonnement qu’elle absorbe est réémis. On a donc d'après la loi de Planck :
\begin{center}
$\Phi_{emis,Terre}$ = $4 \pi R_T^{2} \sigma T^{4}$
\end{center} \vspace{\baselineskip}

De plus, nous avons vu précédemment que
\begin{center}
$\Phi_{emis,Terre}$ = $\Phi_{abs,Terre}$  = 121 $PW$
\end{center} \vspace{\baselineskip}

On peut ainsi retrouver la température de la Terre :
\begin{center}
T = $(\frac{121 \times 10^{15}}{4 \pi R_T^{2} \sigma})^{\frac{1}{4}}$ = $235$ K = -19 °C
\end{center} \vspace{\baselineskip}

On comprend ici que sans atmosphère, la température terrestre
serait de -19°C. Nous pouvons en conclure que négliger l'atmosphère et les phénomènes qui s'y produisent
conduit à des aberrations, puisque la température réelle
sur Terre est bien supérieure à -19°C.

Nous allons dans la partie suivante prendre en compte l'atmosphère terrestre et l'effet de serre qui s'y produit. Nous commencerons par une étude dans le cas du modèle le plus simple, c'est à dire une température constante quelle que soit l'altitude, afin de pouvoir comprendre simplement le phénomène d'effet de serre.


\subsection{Bilan radiatif avec effet de serre}

Désormais, considérons que l’atmosphère et un des gaz qui la compose, le CO2, impactent les flux. En réalité, il s’agit de comprendre que lorsque le flux arrive sur les différentes molécules de CO2, il peut lui arriver 3 destins : soit le flux est transmis, c’est-à-dire qu’il passe à travers les molécules, soit il est absorbé, soit il est réfléchis. Dans notre cas, on néglige la réflexion car elle est très petite devant les 2 autres. Une partie d'un flux est alors transmis, et l’autre est absorbé.

On a :  $\alpha$+$\tau$=1

Ces coefficients d'absorption et de transmission sont propres à chaque molécule et dépendent de la longueur d’onde. Le fait que les coefficients dépendent des longueurs d’onde complexifie le problème. 

Il faut utiliser la notion d'émittance pour calculer le flux que l'on associera à ces coefficients.  

\begin{center}
$\phi$ =$\int M^0_{\lambda,T} \, \mathrm{d}\lambda$
\end{center}

Nous pouvons désormais calculer un flux en prenant en compte les interactions avec les molécules de l’atmosphère.

Nous pouvons présenter ces schémas, que présente les différents flux qui se déroulent entre la Terre, le Soleil et l'atmosphère (pour le bilan radiatif terrestre, voir Figure 1, p8).\vspace{\baselineskip}  

\begin{center}
    \includegraphics[scale=0.35]{Images/schemafluxformules.png}
    Figure 3 : Quantification des flux
\end{center} \vspace{\baselineskip}

$\sigma T_T^4$ = $\frac{1}{2} \int_{\lambda}^{} (1-\tau_{CO_2})M^{0}_{\lambda,T} \, \mathrm{d}\lambda 
                + \bar{a_T} \int_{\lambda}^{} \tau_{CO_2}E^{0}_{\lambda,T} \, \mathrm{d}\lambda$ \vspace{\baselineskip}

L’objectif est de quantifier les différentes flèches, afin d’utiliser le bilan thermique et de déterminer la température de la Terre en considérant les effets des gaz de l’atmosphère.

\noindent Comme nous avons dit ci-dessus, le coefficient d'absorption dépend de la longueur d’onde. 

\noindent On a : 
\begin{center}
    $1-\tau_{CO2}(\lambda)$ = $\int_{0}^{h_{max}} k_{abs}(\lambda) \times n_{CO2}(z) \, \mathrm{d}z$
\end{center}

Nous allons étudier deux cas différents. Dans un premier temps avec une température uniforme dans l'ensemble de l'atmosphère. Puis, avec une température qui dépend de l'altitude. 

On commence par exprimer la quantité de matière de CO2 en fonction de l'altitude.
D'après la loi des Gaz Parfaits, en utilisant le principe fondamental de la statique des fluides on a: 

\begin{center}
    PV=nRT
\end{center}

\begin{center}
    $\Rightarrow n= \frac{PV}{RT}=\frac{N}{Na}$
\end{center}

\begin{center}
    $\Rightarrow\frac{N}{V}= \frac{PNa}{RT}$
\end{center}

\begin{center}
    $\Rightarrow n(z)= \frac{P(z)}{kT} $ 
\end{center}   

avec $k_B$=$\frac{Na}{R}$= constante de Boltzmann \vspace{\baselineskip}

Pour le cas de la température uniforme, on a:

\begin{center}
    $\frac{dP}{dz}$ =$-\rho$g
\end{center}

D'après la loi des Gaz Parfaits: 
\begin{center}
    $\rho$= $\frac{PM}{RT}$
\end{center}

donc 
\begin{center}
    $\frac{dP}{dz}$= $-\frac{PM}{RT}$ g
\end{center}

\begin{center}
    $\Rightarrow\frac{dP}{P}$= $-\frac{gM}{RT}$ dz
\end{center}

\begin{center}
    $\Rightarrow P(z)= P_o e^{-\frac{gM}{RT} z }$
\end{center}

On a donc 
\begin{center}
	$n(z)=\frac{P(z)}{kT}= \frac{ P_o e^{-\frac{gM}{RT} z }}{kT}$
 \end{center}

Pour le deuxième cas, il nous faut des profils de température selon l'altitude.
Ainsi on a les profils suivant:

De 0 à 10 km 
\begin{center}
    $T_o$=288K\\
    $T_f$=219k\\
    $T(z)=-144.53z+41739.13$
\end{center}

De 10 à 19 km 
\begin{center}
    $T_o$=219K\\
    $T_f$=219k\\
    $T(z)=219$
\end{center}

De 19 à 32 km 
\begin{center}
    $T_o$=219K\\
    $T_f$=229k\\
    $T(z)=1300z-26500$
\end{center}

De 32 à 47 km 
\begin{center}
    $T_o$=229K\\
    $T_f$=273k\\
    $T(z)=341z-46068$
\end{center}

\noindent Pour trouver la pression, on utilise une formule plus générale:
\begin{center}
T(z)=$T_o$-az      
\end{center}

\noindent avec a= coefficient directeur, $T_o$ = Température de l'atmosphère initiale et $T_f$ = Température de l'atmosphère finale \vspace{\baselineskip}

On peut à présent calculer la pression en fonction de l'altitude, en utilisant le principe fondamental de la statique des fluides. 
\begin{center}
     $\frac{dP}{dz}$ = $\frac{-\rho Mg}{R(T_o-az)}$ 
\end{center}

\begin{center}
$\Leftrightarrow \frac{dP}{P}$ = $\frac{dz}{(T_o-az)} \times \frac{-\rho Mg}{R}$
\end{center}

\begin{center}
     $\Rightarrow \ln(P)=\frac{Mg}{Ra} ln(T_o-az)+K$\\
\end{center}

\begin{align*}
     \Rightarrow P(z) = e^{\frac{Mg}{Ra}ln(T_o-az)} \times e^k =(T_o -az)^{\frac{Mg}{Ra}} \times C
\end{align*}

avec $e^{k}$ = C \vspace{\baselineskip}

On détermine C :
\begin{center}
$P(0)= P_o = (T_o)^{\frac{Mg}{Ra}} \times C$
\end{center}

\begin{center}
$\Rightarrow C= \frac{P_o}{T_o^{\frac{Mg}{Ra}}}$
\end{center}

\begin{center}
$P(z) = (T_o -az)^{\frac{Mg}{Ra}} \times \frac{P_o}{T_o^{\frac{Mg}{Ra}}}$
\end{center}

\begin{center}
$\Leftrightarrow P(z)= P_o (1-\frac{az}{T_o})^{\frac{Mg}{Ra}}$
\end{center}

On a donc :
\begin{center}
   $n(z)=\frac{P(z)}{kT(z)}= \frac{ P_o (1-\frac{a(z)z}{b(z)})^{\frac{Mg}{Ra(z)}}}{kT(z)}$
\end{center}

On peut ainsi à l'aide du programme Python calculer (1-$\tau_{CO2}$), et ainsi calculer l'intégralité des flèches.

\chapter*{Conclusion et perspectives}
\addcontentsline{toc}{chapter}{Conclusion et perspectives}
\begin{itemize}
\item Conclusions sur le travail réalisé
\item Conclusions sur l'apport personnel de cet E.C. projet
\item Perspectives pour la poursuite de ce projet
\end{itemize}

\vspace{\baselineskip}
pas le temps d'ajouter d'autres gazs et complexifier le modèle

%%% Bibliographie

\newpage

\renewcommand{\notesname}{} % Laisse vide le titre de la section
\chapter*{Bibliographie}
\addcontentsline{toc}{chapter}{Bibliographie}
% \renewcommand{\theendnote}{[\arabic{endnote}]} % Redéfinition de l'affichage des numéros de notes de fin
% \renewcommand{\enoteformat}{[\arabic{endnote}]
% 							\hspace{1em} % Laisse un petit espace entre l'indice et la note
% 							\newline} % Sauter une ligne entre chaque note 
\makeatletter
\renewcommand{\enoteheading}{\par\vspace{1 em}}
\renewcommand{\theenmark}{\makebox[0.5 em][r]{\@theenmark}}
\renewcommand{\enoteformat}{\parindent = 2 em
  							\leftskip = 0.5 em
  							[\theenmark]\enspace\ignorespaces}							
\makeatother
\theendnotes

%%% Annexes

\newpage
\chapter*{Annexes}
\addcontentsline{toc}{chapter}{Annexes}
\renewcommand{\thesection}{\Alph{section}}  % Renomme les futures sections avec lettres majuscules
\setcounter{section}{0} % Réinitialisation numéro (ici lettres) des sections à 0 (respectivement A)

\section{Documentation technique}
\newpage

\section{Listings des programmes réalisés}
\newpage

\section{Schémas de montages, plans de conception...}
\newpage

\section{Propositions de sujets de projets (en lien ou pas avec le projet réalisé)}
\newpage

\section{Mettre du code en annexe}

\end{document}
